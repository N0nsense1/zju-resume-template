\documentclass[10pt]{article}


\setlength{\parindent}{0pt}
\usepackage{xltxtra}
\usepackage{hyperref}
\hypersetup{hidelinks}
\usepackage{url}
\urlstyle{tt}
\usepackage{xcolor}
\definecolor{CVBlue}{RGB}{23,110,191}
\usepackage{calc}
\usepackage{graphicx}
\usepackage{tikz}
\usetikzlibrary{calc}
\usepackage{fontspec}
\usepackage{xeCJK}
\usepackage{enumitem}
\CJKsetecglue{} %% 取消中文与数字之间的间隙


%% 主文档字体设置
\setmainfont[
    Path = fonts/Main/,
    Extension = .otf,
    BoldFont = texgyretermes-bold.otf, % 加粗字体
]{texgyretermes-regular.otf} % 正文字体

% 中文字体设置
\setCJKmainfont[
    Path = fonts/hansans/,
    Extension = .ttf,
    BoldFont = NotoSansSC-Bold.ttf, % 加粗字体
]{NotoSansSC-Regular.otf} % 正文字体


\usepackage{relsize}
\usepackage{xspace}

% 使用 fontawesome(部分图标)
\usepackage{fontawesome} 

% A4纸,上下左右边距
\usepackage[
    a4paper,
    left=1.2cm,
    right=1.2cm,
    top=1.5cm,
    bottom=1cm,
    nohead
]{geometry}

\renewcommand{\baselinestretch}{1.5} % 行间距设为1.5

\usepackage{titlesec}
\usepackage{enumitem}
\setlist{noitemsep} % 取消列表项间的额外间距
%\setlist{nosep} % 取消所有垂直间距
\setlist[itemize]{topsep=0.25em, leftmargin=*}
\setlist[enumerate]{topsep=0.25em, leftmargin=*}

% --- 用于控制【不同项目之间】的垂直距离 ---
\newlength{\interProjectSpacing}
\setlength{\interProjectSpacing}{0.8em} % <--- 在此调整项目之间的距离
\newcommand{\projectsep}{\vspace{\interProjectSpacing}}

% --- 用于控制【项目标题】与下方【项目描述】的距离 ---
\newlength{\intraProjectTitleSep}
\setlength{\intraProjectTitleSep}{0.4em} % <--- 在此调整标题和描述的距离
\newcommand{\titlebreak}{\\[\intraProjectTitleSep]}

% --- 用于控制【项目描述】与下方【要点列表】的距离 ---
\newlength{\intraProjectListTopSep}
\setlength{\intraProjectListTopSep}{0.2em} % <--- 在此调整描述和列表的距离

% =======================================================================


\titleformat{\section}         % 定制 \section 命令 
{\large\bfseries\raggedright} % 将 section 标题设置为大号、粗体且左对齐
{}{0em}                      % 可用于为所有 section 添加前缀(如“章节...”)
{}                           % 可用于在标题前插入代码
[{\color{CVBlue}\titlerule}]  % 在标题后插入一条横线
\titlespacing*{\section}{0cm}{*1.6}{*1.2}



\begin{document}
\pagenumbering{gobble}

  %%%% 利用tikz来定位学校Logo,这里只在第一页显示
\begin{tikzpicture}[remember picture, overlay] 
    \node[anchor = north west] at ($(current page.north west)+(0.5cm,+1.0cm)$) {\includegraphics[height=6cm]{zju.png}};
\end{tikzpicture}%

\centerline{\LARGE\bfseries{徐一啸}} 

\centerline{\normalsize{\faUser\ 3250102927}}
    
\section{\makebox[\widthof{\faGraduationCap}][c]{\color{CVBlue}\faGraduationCap}\ 教育背景}    
\textbf{浙江大学} \hfill 2025.9 -- 至今\\[0.2em] % 标题和正文间加一点距离
大一 \quad GPA:4.45 
\begin{itemize}[nosep]
    \item 现浙江大学超算队(ZJUSCT)成员
\end{itemize}

\projectsep

\textbf{浙江省杭州第二中学} \hfill 2022.9 -- 2025.6\\[0.2em]
2022级13班(竞赛班)毕业生\quad
\begin{itemize}[nosep]
    \item 曾任校学生会干部
\end{itemize}

\section{\makebox[\widthof{\faUsers}][c]{\color{CVBlue}\faUsers}\ 项目经历(部分)}

% --- 第一个项目 ---
\textbf{从零构建的basic编译器} \hfill 2025.07 -- 2025.08 \titlebreak
用python语言从零实现的一个编译器引擎,项目完全没有采用任何第三方库,从零实现了简单的basic语言编译器功能。实现了赋值,多级优先级运算,嵌套循环等功能。
\begin{itemize}[nosep, topsep=\intraProjectListTopSep]
    \item \textbf{标准化编译器组件}: 按标准编译器后端架构实现了Tokenizer(分词器), Parser(语法分析器), Interpreter(解释器)。
    \item \textbf{底层算法实现}: 利用递归下降算法和抽象语法树构造实现多级优先级计算。利用node类高度封装,利于阅读、维护和扩展。
    \item \textbf{嵌套符号表设计}: 利用嵌套符号表实现变量的作用域隔离,循环内定义的变量会在循环后自动销毁。
\end{itemize}

\projectsep

% --- 第二个项目 ---
\textbf{磁性计算体系优化} \hfill 2025.12 -- 至今 \titlebreak
复旦大学计算物质研究所与 ZJUSCT 的合作项目,由我和超算队的另外一位成员合作推进。主要目标是对目前主流的分子动力学模拟推理框架 nequip 和 allegro 进行科学计算优化。
\begin{itemize}[nosep, topsep=\intraProjectListTopSep]
    \item \textbf{环境配置}: 通过 SSH 远程连接校内主机,并用 docker 构建并运行容器。自己修改 makefile 以解决不同依赖之间的版本兼容性问题。
    \item \textbf{程序热点分析}: 通过 deepwiki 了解项目架构和各模块功能,梳理模型结构并挖掘并行性。使用 torch profiler 进行耗时分析。
    \item \textbf{优化尝试}: 调用开源项目的代码,针对矩阵稀疏性和内存传输开销问题采取稀疏计算优化和内核融合操作。尝试在模型推理的关键步骤上用 OpenMP 做并行。
\end{itemize}


\section{\makebox[\widthof{\faInfo}][c]{\color{CVBlue}\faInfo}\ 个人优势}
\begin{itemize}[nosep]
    \item \textbf{项目和竞赛经验}: 高中时参与SJTU的AI应用开发竞赛,积累一定经验;自学OpenMP等计算优化知识,参与HPC校赛获奖,并通过后续面试加入ZJUSCT;目前积极参与队伍项目,积累开发经验。
    \item \textbf{实践与解决问题}: 主动研究硬件配置,独立完成装机与日常维护;善于利用工具优化工作流,如配置代理解决校网和代理冲突问题。
    \item \textbf{自主学习能力}: 假期通过国外高校开源课程自主学习算法、数理基础;主动阅读技术博客,官方文档等,扩大自己的术语库;关注科技前沿进展,通过多种渠道进行了解。
\end{itemize}


\section{\makebox[\widthof{\faCogs}][c]{\color{CVBlue}\faCogs}\ 知识背景}
\begin{itemize}[nosep]
    \item \textbf{编程语言}: Python, C++, C, Bash
    \item \textbf{开发工具}: Git, SSH, Conda, GNUmake, Cmake
    \item \textbf{学习方向}: 科学计算优化:OpenMP, SIMD, CUDA; AI训练\推理加速:Pytorch,分布式训练
    
\end{itemize}

\end{document}
