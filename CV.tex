\documentclass[11pt]{article}


\setlength{\parindent}{0pt}
\usepackage{xltxtra}
\usepackage{hyperref}
\hypersetup{hidelinks}
\usepackage{url}
\urlstyle{tt}
\usepackage{xcolor}
\definecolor{CVBlue}{RGB}{23,110,191}
\usepackage{calc}
\usepackage{graphicx}
\usepackage{tikz}
\usetikzlibrary{calc}
\usepackage{fontspec}
\usepackage{xeCJK}
\usepackage{enumitem}
\CJKsetecglue{} %% 取消中文与数字之间的间隙


%% 主文档字体设置
\setmainfont[
    Path = fonts/Main/,
    Extension = .otf,
    BoldFont = texgyretermes-bold.otf, % 加粗字体
]{texgyretermes-regular.otf} % 正文字体

% 中文字体设置
\setCJKmainfont[
    Path = fonts/hansans/,
    Extension = .ttf,
    BoldFont = NotoSansSC-Bold.ttf, % 加粗字体
]{NotoSansSC-Regular.otf} % 正文字体


\usepackage{relsize}
\usepackage{xspace}

% 使用 fontawesome(部分图标)
\usepackage{fontawesome} 

% A4纸,上下左右边距
\usepackage[
    a4paper,
    left=1.2cm,
    right=1.2cm,
    top=1.5cm,
    bottom=1cm,
    nohead
]{geometry}

\renewcommand{\baselinestretch}{1.5} % 行间距设为1.5

\usepackage{titlesec}
\usepackage{enumitem}
\setlist{noitemsep} % 取消列表项间的额外间距
%\setlist{nosep} % 取消所有垂直间距
\setlist[itemize]{topsep=0.25em, leftmargin=*}
\setlist[enumerate]{topsep=0.25em, leftmargin=*}

% --- 用于控制【不同项目之间】的垂直距离 ---
\newlength{\interProjectSpacing}
\setlength{\interProjectSpacing}{0.9em} % <--- 在此调整项目之间的距离
\newcommand{\projectsep}{\vspace{\interProjectSpacing}}

% --- 用于控制【项目标题】与下方【项目描述】的距离 ---
\newlength{\intraProjectTitleSep}
\setlength{\intraProjectTitleSep}{0.4em} % <--- 在此调整标题和描述的距离
\newcommand{\titlebreak}{\\[\intraProjectTitleSep]}

% --- 用于控制【项目描述】与下方【要点列表】的距离 ---
\newlength{\intraProjectListTopSep}
\setlength{\intraProjectListTopSep}{0.2em} % <--- 在此调整描述和列表的距离

% =======================================================================


\titleformat{\section}         % 定制 \section 命令 
{\large\bfseries\raggedright} % 将 section 标题设置为大号、粗体且左对齐
{}{0em}                      % 可用于为所有 section 添加前缀(如“章节...”)
{}                           % 可用于在标题前插入代码
[{\color{CVBlue}\titlerule}]  % 在标题后插入一条横线
\titlespacing*{\section}{0cm}{*1.6}{*1.2}



\begin{document}
\pagenumbering{gobble}

%%%% 利用tikz来定位照片
\begin{tikzpicture}[remember picture, overlay] 
    \node[anchor = north east] at ($(current page.north east)+(-2cm,-1.2cm)$) {\includegraphics[height=3cm]{avatar.jpg}};
  \end{tikzpicture}%
  %%%% 利用tikz来定位学校Logo,这里只在第一页显示
  \begin{tikzpicture}[remember picture, overlay] 
    \node[anchor = north west] at ($(current page.north west)+(0.5cm,+1.0cm)$) {\includegraphics[height=6cm]{zju.png}};
  \end{tikzpicture}%
\centerline{\LARGE\bfseries{徐一啸}} 

\centerline{\normalsize{\faEnvelopeO\ \href{mailto:3230100000@zju.edu.cn}{3250102927@zju.edu.cn}}} 
    
\section{\makebox[\widthof{\faGraduationCap}][c]{\color{CVBlue}\faGraduationCap}\ 教育背景}    
\textbf{浙江大学} \hfill 2023.9 -- 至今\\[0.2em] % 标题和正文间加一点距离
大一 \quad GPA:4.45 
\begin{itemize}[nosep]
    \item 现浙江大学超算队(ZJUSCT)成员
\end{itemize}

\projectsep

\textbf{浙江省杭州第二中学} \hfill 2022.9 -- 2025.6\\[0.2em]
2022级13班(竞赛班)毕业生\quad
\begin{itemize}[nosep]
    \item 曾任校学生会干部
\end{itemize}

\section{\makebox[\widthof{\faUsers}][c]{\color{CVBlue}\faUsers}\ 项目经历(部分)}

% --- 第一个项目 ---
\textbf{基于Mediapipe的猜拳游戏} \hfill 2024.05 -- 2024.06 \titlebreak
项目描述:项目灵感来源于高中时看到的一篇介绍 mediapipe 的技术博客。项目主要利用周末时间完成,在实现手势识别的基础上,又尝试增加了简易 UI 和联网功能,也是我对个人兴趣的初步探索。
\begin{itemize}[nosep, topsep=\intraProjectListTopSep]
    \item \textbf{数据采集}: \textbf{openCV} 提供了高度封装的视频流采集和转换功能,用笔记本的原生摄像头即可完成图像采集。
    \item \textbf{手势识别}: \textbf{Mediapipe} 提供了预训练好的模型,将图像格式简单转换后传入推理框架即可进行推理。\textbf{Mediapipe} 会识别并输出一组坐标,对应手部各个“关键点”在图像中的位置。项目在一开始初步实现了手势判断逻辑。后续针对识别稳定性和手部大小不同的问题分别作了连续帧投票和手掌比例归一化的优化,在实际测试中表现较好。
    \item \textbf{UI和联网功能}: 利用 \textbf{Gradio} 实现 UI,\textbf{Flask} 实现联网。二者都高度封装,即插即用,为项目提供轻量化实现。
\end{itemize}

\projectsep

% --- 第二个项目 ---
\textbf{磁性计算体系优化} \hfill 2025.12 -- 至今 \titlebreak
项目描述:复旦大学计算物质研究所与 ZJUSCT 的合作项目。主要目标是对目前主流的分子动力学模拟推理框架 nequip 和 allegro 进行科学计算优化和修改,解决大规模磁性原子体系的“维度危机”。我和另一位超算队同学作为主要的参与者负责了目前为止绝大部分的分析和优化工作,并在学期和寒假通过线上会议的形式对接进度和困难点。
\begin{itemize}[nosep, topsep=\intraProjectListTopSep]
    \item \textbf{环境配置}: 通过 SSH 远程连接校内主机,并用 docker 构建并运行容器。自己修改 makefile 以解决不同依赖之间的版本兼容性问题。
    \item \textbf{程序热点分析}: 通过 deepwiki 实现对各模块功能的浏览,梳理模型结构并挖掘并行性。使用 torch profiler 进行耗时分析。
    \item \textbf{优化尝试}: 调用开源项目的代码,针对矩阵稀疏性和内存传输开销问题采取稀疏计算优化和内核融合操作。尝试在模型推理的关键步骤上用 openMPI 做并行。
\end{itemize}


\section{\makebox[\widthof{\faCogs}][c]{\color{CVBlue}\faCogs}\ 技术栈}
\begin{itemize}[nosep]
    \item \textbf{编程语言:} \textbf{Go}, Python, C++
    \item \textbf{开发工具:} SSH, Git, Vim, MakeFile,LaTex
    \item \textbf{操作系统:} Linux
\end{itemize}
\section{\makebox[\widthof{\faGraduationCap}][c]{\color{CVBlue}\faList}\ 获奖情况}
\begin{itemize}
    \item CC98年度用户 \hfill 2023.12
    
\end{itemize}
    
\section{\makebox[\widthof{\faInfo}][c]{\color{CVBlue}\faInfo}\ 其他}
\begin{itemize}[parsep=0.5ex]
    \item \textbf{技术博客:} \href{https://maksymilan.github.io/}{https://maksymilan.github.io/}
    \item \textbf{GitHub:} \href{https://github.com/maksymilan}{https://github.com/maksymilan} 
    \item \textbf{英语水平:} CET-4, CET-6
\end{itemize}
\end{document}
